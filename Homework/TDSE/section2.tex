\section{Appendix:Tight-Binding propagation method and Application}
\paragraph{} 紧束缚(Tight-Binding,TB)方法是一种在凝聚态物理和量子化学中常用的半经验方法,此方法利用原子轨道构建系统的哈密顿量,利用对角化或非对角化方法研究电子结构。\cite{Li}不同于第一性原理计算方法,紧束缚方法中的哈密顿矩阵由经验参数给出,避免了耗时间较长的自洽场计算\cite{sb}。因此紧束缚方法可以处理较大的体系。利用非对角化的TBPM(Tight-Binding propagation method)方法,可以处理含有数十亿个原子轨道的庞大系统。
\subsection{Methodology}
\subsubsection{Tight-binding models}
\qquad 一个包含n个原子轨道的非周期性系统的哈密顿量可以写成如下形式:
\begin{align}
    \hat{H}=\sum_{i}\epsilon_i c^{\dag}_{i}c_{i}-\sum_{i\neq j}t_{ij}c^{\dag}_{i}c_{j}
\end{align}
可以写成如下紧凑的形式:
\begin{equation}
    \begin{aligned}
        &\hat{H}=\mathbf{c^{\dag}Hc}\\
        &\mathbb{c^{\dag}}=[c^{\dag}_{1},c^{\dag}_{2},\cdots,c^{\dag}_{n}]\\
        &H_{ij}=\epsilon_i\delta_{ij}-t_{ij}(1-\delta_{ij})
    \end{aligned}
\end{equation}
其中$\epsilon_i$称为格位积分(on-site integral),$t_{ij}$为跃迁积分。$c^{\dag},c$为产生湮灭算符。格位积分和跃迁积分的定义为:
\begin{equation}
    \begin{aligned}
        \epsilon_i&=\int \psi^{*}_{i}(r)\hat{h_0}\psi_{i}(r)\dd{r}\\
        t_{ij}&=-\int \psi^{*}_{i}(r)\hat{h_0}\psi_{j}(r)\dd{r}
    \end{aligned}
\end{equation}
$\hat{h_0}$是单粒子的哈密顿量:
\begin{equation}
    \hat{h_0}=-\frac{\hbar^2}{2m}\nabla^2+V(r)
\end{equation}
$\psi_i$是单个原子的原子轨道。在实际的计算中,往往采用Wannier函数。
对具有周期性的体系,利用布洛赫定理,我们可以只关注第一个单胞。为此引入一个额外的指标R:
\begin{equation}
    \psi_{iR}(r)=\psi_i(r-R)
\end{equation}
通过傅里叶变换,定义布洛赫波函数和产生湮灭算符:
\begin{equation}
    \begin{aligned}
        \chi_{ik}(r)=\frac{1}{N}\sum_{R}e^{ik\cdot(R+\tau_i)}\psi_{iR}(r)\\
        c^{\dag}_i(k)=\frac{1}{N}\sum_{R}e^{ik\cdot(R+\tau_i)}c^{\dag}_i(R)\\
        c_i(k)=\frac{1}{N}\sum_{R}e^{-ik\cdot(R+\tau_i)}c_i(R)
    \end{aligned}
\end{equation}
其中N表示单胞的数目。Hamiltonian可以写作:
\begin{equation}
    \hat{H}=N\sum_{k}\{ \sum_{i\in uc}\epsilon_ic^{\dag}_{i}(k)c_i(k)-\sum_{R\neq 0,i\neq j}t_{ij}(R)e^{ik\cdot(R+\tau_i-\tau_j)}c^{\dag}_{i}(k)c_j(k)\}
\end{equation}
\subsubsection{Tight-binding propagation method}
对哈密顿矩阵进行对角化,可以精确地得到特征值和特征态,最终得到所有物理量。然而,精确对角化的内存和CPU时间成本随着模型规模N的增长分别为$O(N^2)$和$O(N^3)$,对于大型的模型来说,这种方法是不可行的。相反,TBPM方法使用一种完全不同的方法处理特征值问题,使得内存和时间消耗与N呈线性关系。\\
在TBPM方法中,初始波函数由一组随机生成的状态组成,然后按如下方程演化:
\begin{equation}
    |\varphi(t)\rangle =e^{-i\hat{H}t}|\varphi(0)\rangle
\end{equation}
关联函数中包含哈密顿量的特征,只要演化时间足够长,演化步长足够小,就可以准确捕捉到哈密顿量的全部特性。最后对关联函数求平均值并进行分析,得到物理量。以DOS的关联函数为例,
\begin{equation}
    C_{DOS}(t)=\langle\varphi(0)|\varphi(t)\rangle
\end{equation}
可以证明它和特征值有以下关系:
\begin{equation}
    \langle\varphi(0)|\varphi(t)\rangle=\sum_{ijk}U_{kj}U_{ij}^{*}a_ia^{*}_{k}e^{-\epsilon_jk}
\end{equation}
$\epsilon_j$是第j个特征值。$U_kj$是第j个特征值的第k个分量。初始波函数写作:
\begin{equation}
    |\varphi(0)\rangle=\sum_{i}a_i|\psi_i\rangle
\end{equation}
其中a是符合归一化$\sum_i |a_i|^2=1$的随机复数。$\psi_i$是基矢量。关联函数可以看作是频率为$\epsilon_i$的波的线性组合。利用逆傅里叶变换,可以确定特征值和DOS。\\
为了演化波函数,需要对时间演化算子进行数值展开,由于TB的哈密顿量是稀疏的,所以可以方便地使用Chebyshev多项式进行展开,这种方法对求解含时薛定谔方程是无条件稳定的。假设$x\in[-1,1]$,
\begin{equation}
    e^{-izx}=J_0(z)+2\sum_{m=1}^{\infty}(-i)^mJ_m(z)T_m(x)
\end{equation}
其中$J_m$是m阶贝塞尔函数,$T_m$是第一类切比雪夫多项式。$T_m(x)$可以由递归关系求解:
\begin{equation}
    T_{m+1}(x)=2xT_{m}-T_{m-1}
\end{equation}
为了利用切比雪夫多项式方法,我们需要将$\hat{H}$缩放为$\tilde{H}=\hat{H}\Vert H\Vert$,使$\tilde{H}$的特征值分布在$[-1,1]$。
\begin{equation}
    |\varphi(t)\rangle=\{J_0(\tau)+2\sum_{m=1}^{\infty}(-i)^mJ_m(\tau)\hat{T}_m(\tilde{H})\}|\varphi(0)\rangle
\end{equation}
$\tau=t\cdot\Vert H\Vert$。在实际计算中不需要存储$\hat{T}_m$,而是由递归关系直接计算波函数:
\begin{equation}
    \hat{T}_{m+1}(\tilde{H})\varphi(0)=2\tilde{H}\hat{T}_{m}(\tilde{H})\varphi(0)-\hat{T}_{m-1}(\tilde{H})\varphi(0)
\end{equation}
除了时间演化算子$e^{-it\hat{H}}$外,其他算子也可以使用类似方法,展开为切比雪夫多项式的级数。
\begin{equation}
    f(x)=\frac{1}{2}c_0T_0(x)+\sum_{m}c_mT_m(x)
\end{equation}
系数定义为
\begin{equation}
    c_m=\frac{2}{\pi}\int_{-1}^{1}\frac{\dd{x}}{\sqrt{1-x^2}}f(x)T_m(x)
\end{equation}
令$x=\cos{\theta}$,并带入上式,可以得到:
\begin{equation}
    \begin{aligned}
        c_m=\frac{2}{\pi}\int_{0}^{\pi}f(\cos{\theta})\cos(m\theta)\dd{\theta}\\
        =Re[\frac{2}{\pi}\sum_{n=0}^{N-1}f(\cos\frac{2\pi n}{N})e^{i\frac{2\pi n}{N}m}]
    \end{aligned}
\end{equation}
因此可以使用快速傅里叶变换计算出$c_k$。以Fermi-Dirac算符为例:
\begin{equation}
    \begin{aligned}
        f(\hat{H})&=\frac{ze^{-\beta\hat{H}}}{1+ze^{-\beta\hat{H}}},\\
        \text{where}\quad\beta&=\frac{1}{k_BT},z=e^{\beta\mu}
    \end{aligned}
\end{equation}
我们定义$\tilde{\beta}=\beta\cdot\Vert\hat{H}\Vert$。根据前面的讨论:
\begin{equation}
    f(\tilde{H})=\sum_{m=0}^{\infty}c_mT_m(\tilde{H})
\end{equation}
\subsection{Application}
\subsubsection{Density of states}
通常计算态密度的方法基于精确对角化,在密集的k网格上获得哈密顿矩阵的特征值,并对特征值求和:
\begin{equation}
    D(E)=\sum_{ik}\delta(E-\epsilon_{ik})
\end{equation}
$\epsilon_{ik}$表示k点的第i个特征值。在实际计算中,$\delta$函数往往用高斯分布函数或洛伦兹分布函数近似:
\begin{align}
    G(E-\epsilon_{ik})&=\frac{1}{\sqrt{2\pi}\sigma}\exp{[-\frac{(E-\epsilon_{ik})^2}{2\sigma^2}]}\\
    L(E-\epsilon_{ik})&=\frac{1}{\pi\sigma}\frac{\sigma^2}{(E-\epsilon_{ik})^2+\sigma^2}
\end{align}
TBPM方法则通过对平均关联函数的逆傅里叶变换计算DOS
\begin{equation}
    D(E)=\frac{1}{2\pi S}\sum_{p=1}^{S}\int_{-\infty}^{\infty}C_{DOS}e^{iEt}\dd{t}
    \label{DOS}
\end{equation}
S是随意初始态的数目。式\ref{DOS}可以通过快速傅里叶变换计算,如果需要更高的能量分辨率,也可以使用数值积分计算。TBPM方法计算DOS的误差随系统增大按照$1/\sqrt{SN}$减少,因此,如果想要更高的精度,可以增大模型尺寸或者计算更多的随机态平均。对于足够大的系统,例如$10^8$个轨道,只需要一组随机初始态就可以得到足够精确的结果。关于DOS计算的更多细节,可以参考这篇文章\cite{PhysRevB.82.115448}。
\subsubsection{Local density of states}
为了计算特定轨道i上的LDOS,我们只将$|\varphi(0)\rangle=\sum_{i}a_{i}\psi_i$中的$a_i$设为非零,然后用与DOS相同的方法对关联函数进行分析。
\begin{align}
    d_i(E)&=\sum_{jk}\delta(E-\epsilon_{jk})|U_{ijk}|^2\\
    \langle \varphi(0)|\varphi(t)\rangle&=\sum_{i}|U_{ijk}a_i|^2e^{-i\epsilon_{j}t}
\end{align}
另一种方法基于Lanczos算法\cite{Haydock_1972},利用递归方法在实空间中评估LDOS。特定轨道i上的LDOS是:
\begin{equation}
    d_i(E)=\lim_{\epsilon\rightarrow 0^+}\frac{1}{\pi}\text{Im}\langle \psi_i|G(E+i\epsilon)|\psi_i\rangle
\end{equation}
然后利用迭代的方法计算出格林函数G(E)的对角元素:
\begin{equation}
    \begin{aligned}
        &G_0=\langle l_0|G(E)|l_0\rangle\\
        &=1/\{E-a_0-b_1^2/[E-a_1-b_2^2/(E-a_3-b_3^2\dots)]\}
    \end{aligned}
\end{equation}
式中的$a_n$和$b_n$由以下的迭代关系求出:
\begin{equation}
    \begin{aligned}
        a_i&=\langle l_i|H_i|l_i\rangle\\
        |m_{i+1}\rangle&=(H-a_i)|l_i\rangle-b_i|l_{i-1}\rangle\\
        b_{i+1}&=\sqrt{\langle m_{i+1}|m_{i+1}\rangle}\\
        |l_{i+1}\rangle&=\frac{|m_{i+1}\rangle}{b_{i+1}}\\
        |l_{-1}\rangle&=|0\rangle
    \end{aligned}
\end{equation}
\subsubsection{Optical conductivity}
\qquad 为了计算光导率,TBPM方法将Kubo公式和随机状态方法相结合。Kubo公式通过时间关联函数(time correlation function)计算系统的宏观响应量。它将微观量(如电导率、磁化率)与系统的时间演化联系起来。对于非相互作用的电子系统,由于在$\beta$方向上的场,在$\alpha$方向上的光学电导率的实部为(省略在$\omega=0$处的Drude贡献)
\begin{equation}
    \begin{aligned}
        \text{Re}&=\lim_{E\rightarrow 0^+}\frac{e^{-\beta \hbar \omega}-1}{\hbar\omega A}\int_{0}^{\infty}e^{-Et}sin(\omega t)\\ &\times 2\text{Im}\langle\psi|f(H)J_{\alpha}(t)[1-f(H)]J_{\beta}|\psi\rangle \dd{t}
        \label{sigma}
    \end{aligned}
\end{equation}
A是系统的面积或者体积。对于紧束缚的哈密顿量,流密度算符定义为:
\begin{equation}
    J=-\frac{ie}{\hbar}\sum_{i,j}t_{ij}{\hat{r_j}-\hat{r_i}c^{\dag}_ic_j}
\end{equation}
$\hat{r}$是位置算符。Fermi-Dirac贡献,定义为:
\begin{align}
    f(H)=\frac{1}{e^\beta{H-\mu}-1}
\end{align}
在实际的计算中,通过对实空间的所有基态的随机叠加,确保\ref{sigma}的精度。Fermi-Dirac算符,如前面所述,可以利用切比雪夫多项式和快速傅里叶变换计算。我们引入两个波函数:
\begin{equation}
    \begin{aligned}
        |\psi_1(t)\rangle=e^{-i\tilde{H}t}[1-f(\tilde{H})]J_{\alpha}|\psi(0)\rangle\\
        |\psi_2(t)\rangle=e^{-i\tilde{H}t}f(\tilde{H})|\psi(0)\rangle\\
    \end{aligned}
\end{equation}
于是可得$\sigma_{\alpha\beta}(t)$的实部:
\begin{equation}
    \begin{aligned}
        \text{Re}&=\lim_{E\rightarrow 0^+}\frac{e^{-\beta \hbar \omega}-1}{\hbar\omega A}\int_{0}^{\infty}e^{-Et}sin(\omega t)\\ &\times 2\text{Im}\langle\psi_2|J_{\alpha}(t)|\psi\rangle_{\beta} \dd{t}
        \label{sigma}
    \end{aligned}
\end{equation}
由KK(Kramers-Kronig)关系,推导出$\sigma_{\alpha\beta}$的虚部:
\begin{equation}
    \text{Im}\sigma_{\alpha\beta}(\hbar\omega)=-\frac{1}{\pi}P\int_{-\infty}^{\infty}\frac{\text{Re} \sigma_{\alpha\beta}(\hbar\omega')}{\omega'-\omega}\dd{\omega}
\end{equation}
\subsubsection{DC conductivity}
直流电导率可以通过$\lim \omega\rightarrow0$时的Kubo公式来计算。基于之前的DOS和准本征态计算方法,T=0时,直流电导率在$\alpha$方向上的对角线项为:
\begin{equation}
    \begin{aligned}
        \sigma_{\alpha\alpha}(E)&=\lim_{\tau\rightarrow\infty}\sigma_{\alpha\alpha}(E,\tau)\\
        &=\lim_{\tau\rightarrow\infty}\frac{D(E)}{A}\int_{0}^{\tau}\text{Re}[e^{-iEt}C_{DC}(t)]\dd{t}
    \end{aligned}
\end{equation}
其中DC关联函数定义为:
\begin{equation}
    C_{DC}=\frac{\langle\psi(0)|J_{\alpha}e^{-i\tilde{H}t}J_{\alpha}|\tilde{\Psi}(E)\rangle}{\langle\psi(0)|\tilde{\Psi}(E)\rangle}
\end{equation}
A表示面积或者体积。需要强调的是,$|\psi(0)\rangle$必须与计算$|\tilde{\Psi}(E)\rangle$时所用的随机初始态相同。不考虑安德森局域化影响的半经典直流电导率定义为:
\begin{equation}
    \sigma^{sc}=\sigma^{max}_{\alpha\alpha}(E,\tau)
\end{equation}
实验中测量的场效应载流子迁移率与半经典直流电导率有关:
\begin{equation}
    u(E)=\frac{\sigma^{sc}}{en_{e}(E)}
\end{equation}
其中载流子密度可以由态密度的积分求出:
\begin{equation}
    n_{e}=\int_{0}^{E}D(\varepsilon)\dd{\varepsilon}
\end{equation}
\subsubsection{Diffusion coefficient}
根据Kubo公式,直流电导率也可以写成扩散系数的函数:
\begin{equation}
    \sigma_{\alpha\alpha}(E)=\frac{e^2}{A}D(E)\lim_{\tau\rightarrow\infty}D_{diff}(E,\tau)
\end{equation}
因此,可以反推出扩散系数的计算方法:
\begin{equation}
    D_{diff}=\frac{1}{e^2}\int_{0}^{\tau}e^{-iEt}C_{DC}(t)\dd{t}
\end{equation}
一旦求出了扩散系数,就可以立即得到载流子速率:
\begin{equation}
    v(E)=\sqrt{D_{diff(E,\tau)}/\tau}
\end{equation}
以及平均自由程、安德森局域长度等物理量。
