\documentclass{article}
\usepackage{fullpage}
\usepackage{parskip}
\usepackage{physics}
\usepackage{amssymb}
\usepackage{xcolor}
\usepackage[colorlinks,urlcolor=red]{hyperref}
\usepackage{array}
\usepackage{longtable}
\usepackage{multirow}
\usepackage{ctex}
\setCJKmainfont{FandolSong}

\newcolumntype{M}{>{$\displaystyle}c<{$}}
\newcolumntype{L}{>{$\displaystyle}l<{$}}

\newcommand\Vtextvisiblespace[1][.3em]
{%
	\mbox{\kern.06em\vrule height.3ex}%
	\vbox{\hrule width#1}%
	\hbox{\vrule height.3ex}
}

\newcommand{\cbox}[2][cyan]
{\mathchoice
	{\setlength{\fboxsep}{0pt}\colorbox{#1}{$\displaystyle#2$}}
	{\setlength{\fboxsep}{0pt}\colorbox{#1}{$\textstyle#2$}}
	{\setlength{\fboxsep}{0pt}\colorbox{#1}{$\scriptstyle#2$}}
	{\setlength{\fboxsep}{0pt}\colorbox{#1}{$\scriptscriptstyle#2$}}
}

\newcommand{\typical}{\cbox{\phantom{A}}}
\newcommand{\tall}{\cbox{\phantom{A^{\vphantom{x^x}}_x}}}
\newcommand{\grande}{\cbox{\phantom{\frac{1}{xx}}}}
\newcommand{\venti}{\cbox{\phantom{\sum_x^x}}}

\newcommand{\pkg}[1]{\textsf{#1}}

% physics 1.30
% \title{The \texttt{physics} package}
% \author{Sergio C. de la Barrera \\ \texttt{physics.tex@gmail.com}}
\title{\pkg{physics} 宏包手册}
\author{Sergio C. de la Barrera(著) \\ \texttt{physics.tex@gmail.com}
        \and SwitWu(译) \\ \texttt{2401336502@qq.com}}

\begin{document}
\maketitle

\tableofcontents

% \section{Before you start}
\section{开始之前}

% \subsection{The purpose of this package}
\subsection{宏包设计初衷}

% The goal of this package is to make typesetting equations for physics simpler, faster, and more human-readable. 
% To that end, the commands included in this package have names that make the purpose of each command immediately obvious and
% remove any ambiguity while reading and editing \texttt{physics} code. From a practical standpoint, 
% it is handy to have a well-defined set of shortcuts for accessing the long-form of each of these commands. 
% The commands listed below are therefore defined in terms of their long-form names and then shown explicitly in terms of 
% the default shorthand command sequences. These shorthand commands are meant make it easy to remember both the shorthand names 
% and what each one represents.
此宏包的目标是让物理方程式的排版更加简单、快捷以及具有可读性。为了做到这一点,
此宏包中的命令依照其作用进行命名,并且移除在阅读和编辑 \pkg{physics} 代码时
的任何歧义点。从实践的立场看,给这些命令的“冗长”形式定义合适的缩写集合非常便于
用户使用。因此,下面列出的命令是根据它们的长格式名称定义的,然后根据默认的缩写
命令序列显式展示。这些缩写命令旨在让你轻松记住缩写名称以及每个名称所代表的含义。

% \subsection{Other required packages}
\subsection{其它需要的宏包}

% The \texttt{physics} package requires \texttt{xparse} and \texttt{amsmath} to work properly in your \LaTeX~document. 
% The \texttt{amsmath} package comes standard with most \LaTeX~distributions and is loaded by \texttt{physics} for your convenience.
% You may also already have \texttt{xparse} installed on your system as it is a popular package for defining \LaTeX macros,
% however, if you are unsure you can either install it again using your local package manager (comes with most distributions) or 
% by visiting the \href{http://www.ctan.org/}{CTAN} online package database, or you could even just try to use \texttt{physics} without 
% worrying about it. Many modern \LaTeX compilers will locate and offer to download missing packages for you.
\pkg{physics} 宏包需要 \pkg{xparse} 宏包和 \pkg{amsmath} 宏包的支持来正常工作。
\pkg{amsmath} 标配于大多数 \LaTeX{} 发行版且为了方便已由 \pkg{physics} 宏包进行加载。
考虑到 \pkg{xparse} 是一个用于定义 \LaTeX{} 宏的著名宏包,所以也许你已经在系统上安装了
\pkg{xparse} 宏包。然而,如果你不能确定的话,那么你可以通过使用你的本地宏包管理器\footnote{例如 \TeX{} Live 的宏包管理器为
\textsf{tlmgr},MiK\TeX{} 的宏包管理器为 \textsf{mpm}。}(大多数发行版标配)来进行安装,
也可以通过访问 \href{https://www.ctan.org}{\textsc{ctan}} 在线宏包数据库进行安装,甚至可以尝试
直接使用 \pkg{physics} 宏包。许多现代 \LaTeX{} 编译器会为你定位并提供下载缺失的宏包。

% \subsection{Using \texttt{physics} in your \LaTeX~document}
\subsection{在你的 \LaTeX{} 文档中使用 \pkg{physics} 宏包}
% To use the \texttt{physics} package, simply insert \verb|\usepackage{physics}| in the preamble of your document,
% before \verb|\begin{document}| and after \verb|\documentclass{class}|:
若要使用 \pkg{physics} 宏包,只需在文档导言区插入 \verb|\usepackage{physics}|:
\begin{verbatim}
\documentclass{class}
...
\usepackage{physics}
...
\begin{document}
	content...
\end{document}
\end{verbatim}

% \section{List of commands}
\section{命令列表}

% \subsection{Automatic bracing}
\subsection{自动括号}

\begin{longtable}[l]{ l l p{6cm} }
\verb|\quantity|        & \verb|\qty(\typical)| $\displaystyle\rightarrow \qty(\typical)$ & 自动添加圆括号 $\qty(\;)$ \\
						& \verb|\qty(\tall)| $\displaystyle\rightarrow \qty(\tall)$       &                          \\
						& \verb|\qty(\grande)| $\displaystyle\rightarrow \qty(\grande)$   &                          \\
						& \verb|\qty[\typical]| $\rightarrow \qty[\typical]$              & 自动添加中括号 $\qty[\;]$ \\
						& \verb+\qty|\typical|+ $\rightarrow \qty|\typical|$              & 自动添加竖线 $\qty|\;|$   \\
						& \verb|\qty{\typical}| $\rightarrow \qty{\typical}$              & 自动添加花括号 $\qty{\;}$ \\
						& \verb|\qty\big{}| $\rightarrow \qty\big{}$                      & \multirow{2}{*}{\parbox{6cm}{手动调整尺寸(对上述任意括号类型生效)}} \\
						& \verb|\qty\Big{}| $\rightarrow \qty\Big{}$                      & \\
						& \verb|\qty\bigg{}| $\rightarrow \qty\bigg{}$                    & \\
						& \verb|\qty\Bigg{}| $\rightarrow \qty\Bigg{}$                    & \\
					    & \verb|\pqty{}| $\leftrightarrow$ \verb|\qty()|                   & \multirow{2}{*}{\parbox{6cm}{等价语法,具健壮性且更具 \LaTeX{} 友好性}} \\
					    & \verb|\bqty{}| $\leftrightarrow$ \verb|\qty[]|                   & \\
					    & \verb+\vqty{}+ $\leftrightarrow$ \verb+\qty||+                   & \\
					    & \verb|\Bqty{}| $\leftrightarrow$ \verb|\qty{}|                   & \\
\verb|\absolutevalue|   & \verb|\abs{a}| $\rightarrow \abs{a}$                             & 自动调整尺寸,等价于 \verb|\qty| \!\!\texttt{|a|} \\
                        & \verb|\abs\Big{a}| $\rightarrow \abs\Big{a}$                     & 继承 \verb|\qty| 的手动调整尺寸语法\\
                        & \verb|\abs*{\grande}| $\displaystyle\rightarrow \abs*{\grande}$  & 星号版本用于固定尺寸 \\
\verb|\norm|            & \verb|\norm{a}| $\rightarrow \norm{a}$                           & 自动调整尺寸 \\
                        & \verb|\norm\Big{a}| $\rightarrow \norm\Big{a}$                    & 手动调整尺寸 \\
                        & \verb|\norm*{\grande}| $\displaystyle\rightarrow \norm*{\grande}$ & 星号版本用于固定尺寸 \\
\verb|\evaluated|       & \verb|\eval{x}_0^\infty| $\displaystyle\rightarrow \eval{x}_0^\infty$ & 用于赋值上下限的竖线 \\
                        & \verb|\eval(x| \!\!\texttt{|}\!\! \verb|_0^\infty| $\displaystyle\rightarrow \eval(x|_0^\infty$ & 替代形式 \\
                        & \verb|\eval[x| \!\!\texttt{|}\!\! \verb|_0^\infty| $\displaystyle\rightarrow \eval[x|_0^\infty$ & 替代形式 \\
                        & \verb|\eval[\venti| \!\!\texttt{|}\!\! \verb|_0^\infty| $\displaystyle\rightarrow \eval[\venti|_0^\infty$ & 自动调整尺寸\\
                        & \verb|\eval*[\venti| \!\!\texttt{|}\!\! \verb|_0^\infty| $\displaystyle\rightarrow \eval*[\venti|_0^\infty$ & 星号版本用于固定尺寸 \\
\verb|\order|           & \verb|\order{x^2}| $\rightarrow \order{x^2}$ & 阶;自动处理尺寸与间距 \\
                        & \verb|\order\Big{x^2}| $\rightarrow \order\Big{x^2}$ & 手动调整尺寸 \\
                        & \verb|\order*{\grande}| $\displaystyle\rightarrow \order*{\grande}$ & 星号版本用于固定尺寸 \\
\verb|\commutator|      & \verb|\comm{A}{B}| $\rightarrow \comm{A}{B}$ & 自动调整尺寸 \\
                        & \verb|\comm\Big{A}{B}| $\rightarrow \comm\Big{A}{B}$ & 手动调整尺寸 \\
                        & \verb|\comm*{A}{\grande}| $\displaystyle\rightarrow \comm*{A}{\grande}$ & 星号版本用于固定尺寸 \\
\verb|\anticommutator|  & \verb|\acomm{A}{B}| $\rightarrow \acomm{A}{B}$ & 与 \verb|\poissonbracket| 相同\\
%& \verb|\acommutator{A}{B}| $\rightarrow \acommutator{A}{B}$ & alternate name \\
\verb|\poissonbracket|  & \verb|\pb{A}{B}| $\rightarrow \pb{A}{B}$ & 与 \verb|\anticommutator| 相同
\end{longtable}

% \subsection{Vector notation}
\subsection{向量记号}

% The default del symbol $\vnabla$ used in \texttt{physics} vector notation can be switched to appear with 
% an arrow $\vec{\vnabla}$ by including the option \texttt{arrowdel} in the document preamble 
% $\rightarrow$ \verb|\usepackage[arrowdel]{physics}|.
向量记号中 \href{https://en.wikipedia.org/wiki/Del}{nabla 算符} 默认使用 $\vnabla$,如果需要使用带箭头的 $\vec{\vnabla}$,请在文档导言区添加 \texttt{arrowdel} 选项,
即 \verb|\usepackage[arrowdel]{physics}|。
\begin{longtable}[l]{ l l p{6cm} }
\verb|\vectorbold|  & \verb|\vb{a}| $\rightarrow \vb{a}$ & 直立体/非希腊字母 \\
                    & \verb|\vb*{a}|, \verb|\vb*{\theta}| $\rightarrow \vb*{a}$, $\vb*{\theta}$ & 意大利斜体/希腊字母 \\
\verb|\vectorarrow| & \verb|\va{a}| $\rightarrow \va{a}$ & 直立体/非希腊字母 \\
                    & \verb|\va*{a}|, \verb|\va*{\theta}| $\rightarrow \va*{a}$, $\va*{\theta}$ & 意大利斜体/希腊字母 \\
\verb|\vectorunit|  & \verb|\vu{a}| $\rightarrow \vu{a}$ & 直立体/非希腊字母 \\
                    & \verb|\vu*{a}|, \verb|\vu*{\theta}| $\rightarrow \vu*{a}$, $\vu*{\theta}$ & 意大利斜体/希腊字母 \\
\verb|\dotproduct|  & \verb|\vdot| $\rightarrow \vdot$ 如 $\vb{a} \vdot \vb{b}$ & 注:\verb|\dp| 为受保护的 \TeX{} 原语 \\
\verb|\crossproduct| & \verb|\cross| $\rightarrow \cross$ 如 $\vb{a} \cross \vb{b}$ & 替换名 \\
                     & \verb|\cp| $\rightarrow \cp$ as in $\vb{a} \cp \vb{b}$ & 缩写名称 \\
\verb|\gradient|     & \verb|\grad| $\rightarrow \grad$ & \\
                     & \verb|\grad{\Psi}| $\rightarrow \grad{\Psi}$ & 默认模式 \\
                     & \verb|\grad(\Psi+\tall)| $\displaystyle\rightarrow \grad(\Psi+\tall)$ & 长形式(类似 \verb|\qty|,也处理间距) \\
                     & \verb|\grad[\Psi+\tall]| $\displaystyle\rightarrow \grad[\Psi+\tall]$ & \\
\verb|\divergence|  & \verb|\div| $\rightarrow \div$ & 注:\pkg{amsmath} 中符号 $\divisionsymbol$ 被重命名为 \verb|\divisionsymbol| \\
                    & \verb|\div{\vb{a}}| $\rightarrow \div{\vb{a}}$ & 默认模式 \\
                    & \verb|\div(\vb{a}+\tall)| $\displaystyle\rightarrow \div(\vb{a}+\tall)$ & 长形式 \\
                    & \verb|\div[\vb{a}+\tall]| $\displaystyle\rightarrow \div[\vb{a}+\tall]$ & \\
\verb|\curl|        & \verb|\curl| $\rightarrow \curl$ & \\
                    & \verb|\curl{\vb{a}}| $\rightarrow \curl{\vb{a}}$ & 默认模式 \\
                    & \verb|\curl(\vb{a}+\tall)| $\displaystyle\rightarrow \curl(\vb{a}+\tall)$ & 长形式 \\
                    & \verb|\curl[\vb{a}+\tall]| $\displaystyle\rightarrow \curl[\vb{a}+\tall]$ & \\
\verb|\laplacian|   & \verb|\laplacian| $\rightarrow \laplacian$ & \\
                    & \verb|\laplacian{\Psi}| $\rightarrow \laplacian{\Psi}$ & 默认模式 \\
                    & \verb|\laplacian(\Psi+\tall)| $\displaystyle\rightarrow \laplacian(\Psi+\tall)$ & 长形式 \\
                    & \verb|\laplacian[\Psi+\tall]| $\displaystyle\rightarrow \laplacian[\Psi+\tall]$ &
\end{longtable}

% \subsection{Operators}
\subsection{算符}

% The standard set of trig functions is redefined in \texttt{physics} to provide automatic braces that behave 
% like \verb|\qty()|. In addition, an optional power argument is provided. 
% This behavior can be switched off by including the option \texttt{notrig} 
% in the preamble $\rightarrow$ \verb|\usepackage[notrig]{physics}|.
\pkg{physics} 宏包重定义了标准三角函数集合使得集合中的函数宏可以提供自动调整尺寸的括号。
另外,还提供了一个可选参数用来输出幂。此行为可通过在导言区添加 \texttt{notrig} 选项
进行关闭,即 \begin{verbatim}\usepackage[notrig]{physics}\end{verbatim}

\begin{tabular}[l]{ l l p{8cm} }
\multicolumn{3}{l}{重定义样例:} \\
\verb|\sin| & \verb|\sin(\grande)| $\displaystyle\rightarrow \sin(\grande)$ & 自动添加括号;旧的 \verb|\sin| 重命名为 \verb|\sine| \\
            & \verb|\sin[2](x)| $\rightarrow \sin[2](x)$ & 可选参数输出幂 \\
            & \verb|\sin x| $\rightarrow \sin x$ & 仍然可以无参数使用
\end{tabular}

% The full set of available trig functions in \texttt{physics} includes:
\pkg{physics} 宏包中可用的三角函数集合:

\begin{tabular}{llll}
\verb|\sin(x)| & \verb|\sinh(x)| & \verb|\arcsin(x)| & \verb|\asin(x)| \\
\verb|\cos(x)| & \verb|\cosh(x)| & \verb|\arccos(x)| & \verb|\acos(x)| \\
\verb|\tan(x)| & \verb|\tanh(x)| & \verb|\arctan(x)| & \verb|\atan(x)| \\
\verb|\csc(x)| & \verb|\csch(x)| & \verb|\arccsc(x)| & \verb|\acsc(x)| \\
\verb|\sec(x)| & \verb|\sech(x)| & \verb|\arcsec(x)| & \verb|\asec(x)| \\
\verb|\cot(x)| & \verb|\coth(x)| & \verb|\arccot(x)| & \verb|\acot(x)|
\end{tabular}$\Rightarrow$
\begin{tabular}{MMMM}
\sin(x) & \sinh(x) & \arcsin(x) & \asin(x) \\
\cos(x) & \cosh(x) & \arccos(x) & \acos(x) \\
\tan(x) & \tanh(x) & \arctan(x) & \atan(x) \\
\csc(x) & \csch(x) & \arccsc(x) & \acsc(x) \\
\sec(x) & \sech(x) & \arcsec(x) & \asec(x) \\
\cot(x) & \coth(x) & \arccot(x) & \acot(x)
\end{tabular}

% The standard trig functions (plus a few that are missing in \texttt{amsmath}) are available 
% without any automatic bracing under a new set of longer names:
标准的三角函数(加上 \pkg{amsmath} 中缺少的一些)可以通过如下名称集进行使用,注意这些不会自动调整
括号尺寸。

\begin{tabular}{llll}
\verb|\sine| & \verb|\hypsine| & \verb|\arcsine| & \verb|\asine| \\
\verb|\cosine| & \verb|\hypcosine| & \verb|\arccosine| & \verb|\acosine| \\
\verb|\tangent| & \verb|\hyptangent| & \verb|\arctangent| & \verb|\atangent| \\
\verb|\cosecant| & \verb|\hypcosecant| & \verb|\arccosecant| & \verb|\acosecant| \\
\verb|\secant| & \verb|\hypsecant| & \verb|\arcsecant| & \verb|\asecant| \\
\verb|\cotangent| & \verb|\hypcotangent| & \verb|\arccotangent| & \verb|\acotangent|
\end{tabular}

% Similar behavior has also been extended to the following functions:
类似的行为也被拓展到下列函数:

\begin{tabular}{l>{$}l<{$}ll}
\verb|\exp(\tall)| & \exp(\tall) &                              & \verb|\exponential| \\
\verb|\log(\tall)| & \log(\tall) &                              & \verb|\logarithm| \\
\verb|\ln(\tall)|  & \ln(\tall)  & 旧版本重定义为 $\Rightarrow$  & \verb|\naturallogarithm| \\
\verb|\det(\tall)| & \det(\tall) &                              & \verb|\determinant| \\
\verb|\Pr(\tall)|  & \Pr(\tall)  &                              & \verb|\Probability|
\end{tabular}

\begin{longtable}[l]{ l l p{8cm} }
\multicolumn{3}{l}{新的算符:} \\
\verb|\trace| or \verb|\tr| & \verb|\tr\rho| $\rightarrow \tr\rho$ also \verb|\tr(\tall)| $\rightarrow \tr(\tall)$  & 迹;与三角函数具相同括号功能 \\
\verb|\Trace| or \verb|\Tr| & \verb|\Tr\rho| $\rightarrow \Tr\rho$ & 替换名 \\
\verb|\rank| & \verb|\rank M| $\rightarrow \rank M$ & 矩阵的秩 \\
\verb|\erf| & \verb|\erf(x)|$\rightarrow \erf(x)$ & 高斯误差函数 \\
\verb|\Res| & \verb|\Res[f(z)]|$\rightarrow \Res[f(z)]$ & 留数;与三角函数具相同括号功能 \\
\verb|\principalvalue| & \verb|\pv{\int f(z) \dd{z}}|$\rightarrow \pv{\int f(z) \dd{z}}$ & Cauchy 主值 \\
& \verb|\PV{\int f(z) \dd{z}}|$\rightarrow \PV{\int f(z) \dd{z}}$ & 替换名 \\
\verb|\Re| & \verb|\Re{z}| $\rightarrow \Re{z}$ & 旧版 \verb|\Re| 重命名为 \verb|\real| $\rightarrow \real$ \\
\verb|\Im| & \verb|\Im{z}| $\rightarrow \Im{z}$ & 旧版 \verb|\Im| 重命名为 \verb|\imaginary| $\rightarrow \imaginary$
\end{longtable}

% \subsection{Quick quad text}
\subsection{快速添加带铅空文本}

% This set of commands produces text in math-mode padded by \verb|\quad| spacing on either side. 
% This is meant to provide a quick way to insert simple words or phrases in a sequence of equations. 
% Each of the following commands includes a starred version which pads the text only on the right side 
% with \verb|\quad| for use in aligned environments such as \texttt{cases}.
这组命令在数学模式中提供两端带 \verb|\quad| 间距的文本,为在方程中插入简单的词语及短语
提供了快捷的方式。下面每一个命令都有一个星号版本用于只在右侧填充 \verb|\quad| 间距,这样方便用于
像 \texttt{cases} 这样的对齐环境。

\begin{tabular}[l]{ l l p{6cm} }
一般性文本:    &              & \\
\verb|\qqtext| & \verb|\qq{}| & 单参数 \\
               & \verb|\qq{word or phrase}| $\rightarrow$\Vtextvisiblespace[1em]$\text{word or phrase}$\Vtextvisiblespace[1em] & 一般模式;两端都填充 \verb|\quad| 间距\\
               & \verb|\qq*{word or phrase}| $\rightarrow \text{word or phrase}$\Vtextvisiblespace[1em] & 星号版本;只在右侧填充 \verb|\quad| 间距
\end{tabular}

\begin{longtable}[l]{ l l }
特殊宏: & \\
\verb|\qcomma| 或者 \verb|\qc| $\rightarrow ,$\Vtextvisiblespace[1em] & 仅右侧填充 \verb|\quad| 间距 \\
\verb|\qcc| $\rightarrow$\Vtextvisiblespace[1em]$\text{c.c.}$\Vtextvisiblespace[1em] & 复共轭;两端都填充 \verb|\quad| 间距除非使用星号版本 \verb|\qcc*| $\rightarrow \text{c.c.}$\Vtextvisiblespace[1em] \\
\verb|\qif| $\rightarrow$\Vtextvisiblespace[1em]$\text{if}$\Vtextvisiblespace[1em] & 两端都填充 \verb|\quad| 间距除非使用星号版本 \verb|\qif*| $\rightarrow \text{if}$\Vtextvisiblespace[1em]
\end{longtable}
\begin{longtable}[l]{ l }
类似于 \verb|\qif| 的还有: \\
\verb|\qthen|, \verb|\qelse|, \verb|\qotherwise|, \verb|\qunless|, \verb|\qgiven|, \verb|\qusing|, \verb|\qassume|, \verb|\qsince|, \\
\verb|\qlet|, \verb|\qfor|, \verb|\qall|, \verb|\qeven|, \verb|\qodd|, \verb|\qinteger|, \verb|\qand|, \verb|\qor|, \verb|\qas|, \verb|\qin|
\end{longtable}

% \subsection{Derivatives}
\subsection{导数}

% The default differential symbol $\dd$ which is used in \verb|\differential| 
% and \verb|\derivative| can be switched to an italic form $d$ by including the 
% option \texttt{italicdiff} in the preamble $\rightarrow$ \verb|\usepackage[italicdiff]{physics}|.
如果需要将在 \verb|\differential| 和 \verb|\derivative| 中使用的默认微分符号 $\dd$ 
切换为意大利形式的 $d$,请在文档导言区添加 \texttt{italicdiff} 选项,即 \verb|\usepackage[italicdiff]{physics}|。

\begin{longtable}[l]{ l l p{6cm} }
\verb|\differential| & \verb|\dd| $\rightarrow \dd$ & \\
& \verb|\dd x| $\rightarrow \dd x$ & 无间距(不推荐) \\
& \verb|\dd{x}| $\rightarrow$ \textvisiblespace\,$\dd{x}$\textvisiblespace & 基于邻接项自动添加间距 \\
& \verb|\dd[3]{x}| $\rightarrow \dd[3]{x}$ & 可选参数输出幂 \\
& \verb|\dd(\cos\theta)| $\rightarrow \dd(\cos\theta)$ & 长形式;自动调整括号尺寸 \\
\verb|\derivative| & \verb|\dv{x}| $\displaystyle\rightarrow \dv{x}$ & 单参数 \\
& \verb|\dv{f}{x}| $\displaystyle\rightarrow \dv{f}{x}$ & 双参数 \\
& \verb|\dv[n]{f}{x}| $\displaystyle\rightarrow \dv[n]{f}{x}$ & 可选参数输出幂 \\
& \verb|\dv{x}(\grande)| $\displaystyle\rightarrow \dv{x}(\grande)$ & 长形式;自动调整括号尺寸和间距 \\
& \verb|\dv*{f}{x}| $\displaystyle\rightarrow \dv*{f}{x}$ & 使用 \verb|\flatfrac| 的行内形式 \\
\verb|\partialderivative| & \verb|\pderivative{x}| $\displaystyle\rightarrow \pderivative{x}$ & 替换名 \\
& \verb|\pdv{x}| $\displaystyle\rightarrow \pdv{x}$ & 缩写名 \\
& \verb|\pdv{f}{x}| $\displaystyle\rightarrow \pdv{f}{x}$ & 双参数 \\
& \verb|\pdv[n]{f}{x}| $\displaystyle\rightarrow \pdv[n]{f}{x}$ & 可选参数输出幂 \\
& \verb|\pdv{x}(\grande)| $\displaystyle\rightarrow \pdv{x}(\grande)$ & 长形式 \\
& \verb|\pdv{f}{x}{y}| $\displaystyle\rightarrow \pdv{f}{x}{y}$ & 混合偏导数 \\
& \verb|\pdv*{f}{x}| $\displaystyle\rightarrow \pdv*{f}{x}$ & 使用 \verb|\flatfrac| 的行内形式\\
\verb|\variation| & \verb|\var{F[g(x)]}| $\rightarrow \var{F[g(x)]}$ & 泛函变体(机制类似于 \verb|\dd|) \\
& \verb|\var(E-TS)| $\rightarrow \var(E-TS)$ & 长形式 \\
\verb|\functionalderivative| & \verb|\fdv{g}| $\displaystyle\rightarrow \fdv{g}$ & 泛函导数(机制类似于 \verb|\dv|) \\
& \verb|\fdv{F}{g}| $\displaystyle\rightarrow \fdv{F}{g}$ & \\
& \verb|\fdv{V}(E-TS)| $\displaystyle\rightarrow \fdv{V}(E-TS)$ & 长形式 \\
& \verb|\fdv*{F}{x}| $\displaystyle\rightarrow \fdv*{F}{x}$ & 使用 \verb|\flatfrac| 的行内形式
\end{longtable}

% \subsection{Dirac bra-ket notation}
\subsection{Dirac bra-ket 记号}
% The following collection of macros for Dirac notation contains two fundamental commands, 
% \verb|\bra| and \verb|\ket|, along with a set of more specialized macros which are essentially 
% combinations of the fundamental pair. The specialized macros are both useful and descriptive 
% from the perspective of generating \texttt{physics} code, however, the fundamental commands 
% are designed to contract with one another algebraically when appropriate and are thus suggested 
% for general use. For instance, the following code renders correctly\footnote{Note the lack of a 
% space between the bra and ket commands. This is necessary is order for the bra to find the 
% corresponding ket and form a contraction.}
下面关于 \href{https://en.wikipedia.org/wiki/Bra%E2%80%93ket_notation}{Dirac 记号}的宏集合包含两个基本的命令,
\verb|\bra| 和 \verb|\ket|,此外还有一系列由两个基本命令组合而成的更加专门的宏。
从生成 \pkg{physics} 代码的角度来看,专用宏既有用又具有描述性,但是,两个基本命令旨在在适当的时候
相互代数性结合,因此建议将其一般性使用。例如,下面的代码表达正确:
\begin{displaymath}
\verb|\bra{\phi}\ket{\psi}| \rightarrow \bra{\phi}\ket{\psi} \qq{as opposed to} \bra{\phi} \ket{\psi}
\end{displaymath}
% whereas a similar construction with higher-level macros will not contract in a robust manner
而具有更高级别宏的类似结构不会以稳健的方式收缩
\begin{displaymath}
\verb|\bra{\phi}\dyad{\psi}{\xi}| \rightarrow \bra{\phi}\dyad{\psi}{\xi}.
\end{displaymath}
% On the other hand, the correct output can be generated by sticking to the fundamental commands,
另一方面,可以通过坚持基本命令来生成正确的输出,
\begin{displaymath}
\verb|\bra{\phi}\ket{\psi}\bra{\xi}| \rightarrow \bra{\phi}\ket{\psi}\bra{\xi}
\end{displaymath}
% allowing the user to type out complicated quantum mechanical expressions without worrying 
% about bra-ket contractions. That being said, the high-level macros do have a place in 
% convenience and readability, as long as the user is aware of rendering issues that may 
% arise due to an absence of automatic contractions.
允许用户排版出复杂的量子力学表达式而不必担心 bra-ket 收缩。
话虽如此,高级宏确实在便利性和可读性方面占有一席之地,只要用户意识到由于缺乏自动收缩而可能出现的渲染问题。
\begin{longtable}[l]{ l L p{6cm} }
\verb|\ket| & \verb|\ket{\tall}| \rightarrow \ket{\tall} & 自动调整尺寸 \\
& \verb|\ket*{\tall}| \rightarrow \ket*{\tall} & 固定尺寸 \\
\verb|\bra| & \verb|\bra{\tall}| \rightarrow \bra{\tall} & 自动调整尺寸 \\
& \verb|\bra*{\tall}| \rightarrow \bra*{\tall} & 固定尺寸 \\
& \verb|\bra{\phi}\ket{\psi}| \rightarrow \bra{\phi}\ket{\psi} & 自动收缩 \\
& \verb|\bra{\phi}\ket{\tall}| \rightarrow \bra{\phi}\ket{\tall} & 自动收缩和调整尺寸 \\
& \verb|\bra{\phi}\ket*{\tall}| \rightarrow \bra{\phi}\ket*{\tall} & \multirow{2}{*}{\parbox{6cm}{任一项带星号都禁止调整尺寸}} \\
& \verb|\bra*{\phi}\ket{\tall}| \rightarrow \bra*{\phi}\ket{\tall} & \\
& \verb|\bra*{\phi}\ket*{\tall}| \rightarrow \bra*{\phi}\ket*{\tall} & \\
\verb|\innerproduct| & \verb|\braket{a}{b}| \rightarrow \braket{a}{b} & 双参数 braket \\
& \verb|\braket{a}| \rightarrow \braket{a} & 单参数(范数) \\
& \verb|\braket{a}{\tall}| \rightarrow \braket{a}{\tall} & 自动调整尺寸 \\
& \verb|\braket*{a}{\tall}| \rightarrow \braket*{a}{\tall} & 固定尺寸 \\
& \verb|\ip{a}{b}| \rightarrow \ip{a}{b} & 缩写名 \\
\verb|\outerproduct| & \verb|\dyad{a}{b}| \rightarrow \dyad{a}{b} & 双参数 dyad \\
& \verb|\dyad{a}| \rightarrow \dyad{a} & 单参数 (projector) \\
& \verb|\dyad{a}{\tall}| \rightarrow \dyad{a}{\tall} & 自动调整尺寸 \\
& \verb|\dyad*{a}{\tall}| \rightarrow \dyad*{a}{\tall} & 固定尺寸 \\
& \verb|\ketbra{a}{b}| \rightarrow \ketbra{a}{b} & 替换名 \\
& \verb|\op{a}{b}| \rightarrow \op{a}{b} & 缩写名 \\
\verb|\expectationvalue| & \verb|\expval{A}| \rightarrow \expval{A} & 隐式形式 \\
& \verb|\expval{A}{\Psi}| \rightarrow \expval{A}{\Psi} & 显式形式 \\
& \verb|\ev{A}{\Psi}| \rightarrow \ev{A}{\Psi} & 缩写名 \\
& \verb|\ev{\grande}{\Psi}| \rightarrow \ev{\grande}{\Psi} & 默认尺寸忽略中间参数 \\
& \verb|\ev*{\grande}{\tall}| \rightarrow \ev*{\grande}{\tall} & 单星号版本固定尺寸 \\
& \verb|\ev**{\grande}{\Psi}| \rightarrow \ev**{\grande}{\Psi} & 双星号版本基于所有部分调整尺寸 \\
\verb|\matrixelement| & \verb|\matrixel{n}{A}{m}| \rightarrow \matrixel{n}{A}{m} & 三个参数都必需 \\
& \verb|\mel{n}{A}{m}| \rightarrow \mel{n}{A}{m} & 缩写名 \\
& \verb|\mel{n}{\grande}{m}| \rightarrow \mel{n}{\grande}{m} & 默认尺寸忽略中间参数 \\
& \verb|\mel*{n}{\grande}{\tall}| \rightarrow \mel*{n}{\grande}{\tall} & 单星号版本固定尺寸 \\
& \verb|\mel**{n}{\grande}{m}| \rightarrow \mel**{n}{\grande}{m} & 双星号版本基于所有部分调整尺寸
\end{longtable}

% \subsection{Matrix macros}
\subsection{矩阵宏}

% The following matrix macros produce unformatted rows and columns of matrix elements for use as separate matrices 
% as well as blocks within larger matrices. For example, the command \verb|\identitymatrix{2}| which has also has 
% the shortcut \verb|\imat{2}| produces the elements of a $2 \times 2$ identity matrix $\smqty{\imat{2}}$ without 
% braces or grouping. This allows the command to also be used within another matrix, as in:
下列矩阵宏生成由矩阵元组成的未格式化的行与列,其既可以单独使用,也可以充当大型矩阵中的块。
例如,命令 \verb|\identitymatrix{2}|(其有缩写形式 \verb|\imat{2}|)用于生成一个
不带括号及分组的 $2\times 2$ 单位矩阵 $\smqty{\imat{2}}$,这使得该命令可以在另一个
矩阵之中使用,例如:

\begin{minipage}{3cm}
\begin{verbatim}
\begin{pmatrix}
\imat{2} \\ a & b
\end{pmatrix}
\end{verbatim}
\end{minipage}
\begin{minipage}{6cm}
\begin{displaymath}
\Rightarrow\qquad
\begin{pmatrix}
\imat{2} \\ a & b
\end{pmatrix}
\end{displaymath}
\end{minipage}

% To specify elements on the right of left sides of our \verb|\imat{2}| sub-matrix we use the grouping command 
% \verb|\matrixquantity| or \verb|\mqty| to effectively convert \verb|\imat{2}| into a single matrix element of 
% a larger matrix:
为了指定 \verb|\imat{2}| 子矩阵右侧的元,我们使用分组命令 \verb|\matrixquantity| 或者 \verb|\mqty| 
来将 \verb|\imat{2}| 转化为大矩阵中的一个单矩阵元。

\begin{minipage}{9cm}
\begin{verbatim}
\begin{pmatrix}
\mqty{\imat{2}} & \mqty{a\\b} \\ \mqty{c & d} & e
\end{pmatrix}
\end{verbatim}
\end{minipage}
\begin{minipage}{6cm}
\begin{displaymath}
\Rightarrow\qquad
\begin{pmatrix}
\mqty{\imat{2}} & \mqty{a\\b} \\ \mqty{c & d} & e
\end{pmatrix}
\end{displaymath}
\end{minipage}

% The extra \verb|\mqty| groups were required in this case in order to get the $a$ and $b$ elements to 
% behave as a single element, since \verb|\mqty{\imat{2}}| also acts like a single matrix element 
% (the same can be said of the grouped $c$ and $d$ elements). Finally, the outermost \texttt{pmatrix} 
% environment could have also been replaced with the \texttt{physics} macro \verb|\mqty()|, 
% allowing the above example to be written on one line:
在这个情形下需要额外的 \verb|\mqty| 分组来把 $a$ 和 $b$ 看作一个单元,这是因为
\verb|\mqty{\imat{2}}| 也被视作一个单矩阵元($c$ 和 $d$ 的分组同理说明)。
最后,最外层的 \texttt{pmatrix} 环境也可以替换为 \pkg{physics} 宏 \verb|\mqty()|,
从而使得上面的例子通过一行代码完成:

\begin{minipage}{10cm}
\begin{verbatim}
\mqty(\mqty{\imat{2}} & \mqty{a\\b} \\ \mqty{c & d} & e)
\end{verbatim}
\end{minipage}
\begin{minipage}{6cm}
\begin{displaymath}
\Rightarrow\qquad
\mqty(\mqty{\imat{2}} & \mqty{a\\b} \\ \mqty{c & d} & e)
\end{displaymath}
\end{minipage}

\begin{longtable}[l]{ l L p{6cm} }
\verb|\matrixquantity| & \verb|\mqty{a & b \\ c & d}| \rightarrow \mqty{a & b \\ c & d} & 将矩阵元素集合组为一个单位体 \\
& \verb|\mqty(a & b \\ c & d)| \rightarrow {\mqty(a & b \\ c & d)} & 圆括号 \\
& \verb|\mqty*(a & b \\ c & d)| \rightarrow {\mqty*(a & b \\ c & d)} & 替换圆括号 \\
& \verb|\mqty[a & b \\ c & d]| \rightarrow {\mqty[a & b \\ c & d]} & 中括号 \\
& \verb|\mqty| \texttt{|} \verb|a & b \\ c & d| \texttt{|} \rightarrow {\mqty|a & b \\ c & d|} & 垂直竖线 \\
& \verb|\pmqty{}| \leftrightarrow \verb|\mqty()| & \multirow{2}{*}{\parbox{6cm}{等价语法;具健壮性且更具 \LaTeX{} 友好性}} \\
& \verb|\Pmqty{}| \leftrightarrow \verb|\mqty*()| & \\
& \verb|\bmqty{}| \leftrightarrow \verb|\mqty[]| & \\
& \verb|\vmqty{}| \leftrightarrow \verb+\mqty||+ & \\
\verb|\smallmatrixquantity| & \verb|\smqty{a & b \\ c & d}| \rightarrow \smqty{a & b \\ c & d} & \verb|\mqty| 的 \texttt{smallmatrix} 形式 \\
& \verb|\smqty()| \qor \verb|\spmqty{}| & \verb|\mqty()| 的“小”版本 \\
& \verb|\smqty*()| \qor \verb|\sPmqty{}| & \verb|\mqty*()| 的“小”版本 \\
& \verb|\smqty[]| \qor \verb|\sbmqty{}| & \verb|\mqty[]| 的“小”版本 \\
& \verb+\smqty||+ \qor \verb|\svmqty{}| & \verb+\mqty||+ 的“小”版本 \\
\verb|\matrixdeterminant| & \verb|\mdet{a & b \\ c & d}| \rightarrow {\mdet{a & b \\ c & d}} & 矩阵行列式 \\
& \verb|\smdet{a & b \\ c & d}| \rightarrow {\smdet{a & b \\ c & d}} & 小型矩阵行列式 \\
\verb|\identitymatrix| & \verb|\imat{n}| & $n \times n$ 单位矩阵的元 \\
& \verb|\smqty(\imat{3})| \rightarrow \smqty(\imat{3}) & 使用 \verb|\mqty| 或者 \verb|\smqty| 进行格式化 \\
\verb|\xmatrix| & \verb|\xmat{x}{n}{m}| & 以 $x$ 填充的 $n \times m$ 矩阵的元 \\
& \verb|\smqty(\xmat{1}{2}{3})| \rightarrow \smqty(\xmat{1}{2}{3}) & 使用 \verb|\mqty| 或者 \verb|\smqty| 进行格式化 \\
& \verb|\smqty(\xmat*{a}{3}{3})| \rightarrow \smqty(\xmat*{a}{3}{3}) & 星号版本生成元素指标 \\
& \verb|\smqty(\xmat*{a}{3}{1})| \rightarrow \smqty(\xmat*{a}{3}{1}) & 带指标向量 \\
& \verb|\smqty(\xmat*{a}{1}{3})| \rightarrow \smqty(\xmat*{a}{1}{3}) & \\
\verb|\zeromatrix| & \verb|\zmat{n}{m}| & 以 $0$ 填充的 $n \times m$ 矩阵 \\
& \verb|\smqty(\zmat{2}{2})| \rightarrow \smqty(\zmat{2}{2}) & 等价于 \verb|\xmat{0}{n}{m}| \\
\verb|\paulimatrix| & \verb|\pmat{n}| & 第 $n$ 个\href{https://en.wikipedia.org/wiki/Pauli_matrices}{泡利矩阵} \\
& \verb|\smqty(\pmat{0})| \rightarrow \smqty(\pmat{0}) & $n\in \lbrace 0,1,2,3$ 或者 $x,y,z \rbrace$ \\
& \verb|\smqty(\pmat{1})| \rightarrow \smqty(\pmat{1}) & \\
& \verb|\smqty(\pmat{2})| \rightarrow \smqty(\pmat{2}) & \\
& \verb|\smqty(\pmat{3})| \rightarrow \smqty(\pmat{3}) & \\
\verb|\diagonalmatrix| & \verb|\dmat{a,b,c,...}| & \multirow{2}{*}{\parbox{6cm}{最多指定八个对角或块对角元}} \\
& \verb|\mqty(\dmat{1,2,3})| \rightarrow \mqty(\dmat{1,2,3}) & \\
& \verb|\mqty(\dmat[0]{1,2})| \rightarrow \mqty(\dmat[0]{1,2}) & 可选参数填充非对角部分 \\
& \verb|\mqty(\dmat{1,2&3\\4&5})| \rightarrow \mqty(\dmat{1,2&3\\4&5}) & \parbox{6cm}{为每块输入矩阵元使其作为一个单独的对角元} \\
\verb|\antidiagonalmatrix| & \verb|\admat{a,b,c,...}| & 语法同 \verb|\dmat| \\
& \verb|\mqty(\admat{1,2,3})| \rightarrow \mqty(\admat{1,2,3}) & \\
\end{longtable}

\end{document}